\section{Discussions and future work}
\label{sec:conclude}
%{\em\color{gray} Need pargarph that resume what was presented}

As autonomous agents become more and more flexible it becomes difficult
to deal with all the possible changes that occur during the
mission. During our work we identified that, while many existing 
architectures embedding a planner do provide solutions to accept goals
on the fly, the way they handle the execution of the plan beforehand
is not necessarily fitted for efficient integration of such unpredicted
situation. Until now the only way we could improve this was to engineer
extra information within the domain model which added complexity both in
term of the planning search and more importantly in the engineering fo
such model. This added complexity is increasing the risks of modeling
error and for this reason we did provide a simple but yet efficient
solution that allow to propagate the information based on an extended
goal specification. 


The current work is focusing on the plan  structure at a fairly
abstract level without direct consideration in our approach of the
timing constraints applied to each goals of the plan. In this approach
we are able using a breadth first search within the existing plan
structure to select different execution policies for each action
within the plan. By doing so we can improve the overall mission 
execution as it avoids to be proactive on objectives that could lead it
to inefficient handling of new goal requests.

At this stage our causal link traversal is agnostic of the temporal 
constraints applied to the timepoints of the plan. While it is
beneficial for the generality of the approach, we also see that
including such information could greatly the quality of our solution. 
One specific point we see is that often the goals we necessarily mark
as non urgent are tied to a timepoint that is constrained to be {\em
  after} a future date. We need to explore this further in order to
see how such information could be used for our system to automatically
classify such goal as non urgent. It can also impact  also how the
{\em urgent} goal propagate within the plan. For example in our result
we had a non urgent goal around $1:00 pm$ that was eventually 
followed by an urgent goal as a result of the planning. As of today 
our policy did make all the actions related to the non urgent goal 
become proactive. This can lead to premature surfacing that is then 
prohibiting the agent to accept goals in the morning. our current
understanding is that this timing constraint enforcing the non
urgent {\em Surface} to end after $1:00$ is acting as a guard on the
propagation of our heuristic. Therefore we should have ideally only
marked as urgent the actions that are directly after this timepoint
instead of continue to propagate to the actions leading to the
surface. 

Such refinement can be done only by considering the temporal
constraints applied to the plan timepoints. Indeed should the 
{\em Surface} not have been constrained to be around $1:00 pm$ staring
its execution proactively would have not blocked the vehicle at the
surface. We plan to explore in the future to explore how both the
marking of the goals {\em as urgent} and propagation to action  can be
better deduced by analyzing the simple temporal network (STN)
supporting the current plan. Further, as the planner we used did not
implement STNU and the work presented in \cite{morris01} about dynamic
controllability. We believe that as this work is handled during
planning while our approach is handled during plan execution, both
should be easily complementary still the {\bf wait} actions inserted
by their algorithm may need to be considered with a non
urgent status within our approach. 

All of these refinements are considered in order to further improve
our approach int the context of temporal planning. They would allow to
further improve the overal execution of our agent despite the fact
that the set of goals it will receive is not fully known until mission 
completion. In our domain such requirement became crucial as our $AUV$
is evolving on a highly dynamic environment and scientists often need
fresh information in ordert to identify what could be done next. Until
now the common approach on AUV surveys was to separe exploration and
exploitation into two different surveys \cite{Yoerger01012007}. We do
believe that by improving how the vehicle can handle the reception of
new goals can  greatly improve the efficiency of such surveys and can
also generalized to any domain where a robot can receive unexpected
new objectives from its users during the course  of its mission.





% While there has already been work around the topic of robust
% \fcomment{This is just a quick presentation on potential future
%   direction} As we stated in the related works our approach does not
% really address dynamic controllability and has the more classic
% assumption present in many planning frameworks that time-points are
% controllable. A side effect of this is that in its current state it
% may result on the system to decide to defer action as late as
% possible. In our example, this would result on the AUV leaving Vent1
% as late as 19:00 making the rest of its plan brittle to any delay due
% for example to downward water currents on its way to the surface. This
% needs to be further addressed in the future and, especially, how our
% work can be integrated with work presented in \cite{morris01}.

% Further as of today, we consider that the qualification of the goal is
% predefined when the goal is submitted by the planner. It is possible
% though that part of this can be refined on some cases based on the
% nature of the goal. Looking back at our domain, one can note that the
% 2 goals provided are constrained differently on their start time;
% while {\em Sample Vent2} start time is limited only on its upper
% bound, the returning to surface conversely is constrained only on the
% lower bound of its start time. This difference hints on some of the
% issues we presented. While we do consider that explicit information of
% these goals help the plan execution to be improved when such
% information is not initially present. We also are aware that the
% nature of the constraints within the goal itself can help identify the
% best policy to be done. It is obvious that for Sampling Vent2 it is
% better to be proactive on the actions that contribute to this
% goal. Conversely, the other goal only matter if it appears fairly late
% in the plan which means that it is probably better to not start
% completing this part of the plan too aggressively. We plan to further
% explore how we can refine the distinction between the different
% policies by using the information provided by the constraints of the
% different objectives.
 
%%% Local Variables: 
%%% mode: latex
%%% TeX-master: "aaai13"
%%% End: 

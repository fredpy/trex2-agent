\section{Introduction}
\label{sec:intro}

As autonomous robots become more robust and persistent the likelihood
of their receiving new or modified objectives for tasking, during an
ongoing mission is high. Generative planning integrated within the
robotic architecture to fulfill current or such evolving agent
objectives and to deal with unanticipated situations is one approach
to enhance persistence while dealing with robot adaptation.  Earlier
work such as \texttt{IPEM} \cite{AmbrosIngerson88}, \texttt{ROGUE}
\cite{Haigh98}, and the LAAS architecture \cite{alami:1998p820} have
all contributed to this field of research. Results from such work have
lead to goal directed architectures that have sucessfully worked in
the real-world such as the Remote Agent Experiment \cite{mus98}, the
Autonomous Spacecraft Experiment \cite{chien99} and more recently the
Teleo Reactive Executive \cite{py10}. Often these architectures embed
planning engines that support a rich representation dealing with
durative actions and resources while ensuring partial plans are
flexible for robust execution \cite{lemai04}.  A resulting consequence
has been results in efficient approaches to temporal plan dispatch to
ensure robust and sound execution \cite{mus98a,morris01}. 

While these approaches have been necessary to correctly execute a
\emph{flexible temporal plan}, they view the agent itself as
synchronously fulfilling goals provided a priori, without the
introduction of new goals which could alter mission state and
execution.
% This oversight is understandable as such goals are by nature unknown
% and therefore impossible to integrate within the plan limiting the
% ability to formally take them into account.
The natural choice was to make the agent execute its actions as early
as the temporal plan and the current context would allow, with the
assumption that, the robot would finish its plan as early as possible
providing more 'temporal' room for potential future objectives.  This
can be problematic when new objectives are added during the course of
mission execution and especially pronounced with persistent
robots. For instance, the agent may prematurely task the robot to
leave an area of exploration (and exploitation) before an oncoming new
objective for this same geographical area is received resulting in
superfluous actions from the robot. Conversely, the robot waiting
inordinately could waste time with consequent lack of flexibility for
coping with diverse execution scenarios. In this context, it is
important for the agent to make a distinction between actions that can
be taken proactively versus other actions that may not be urgent.

In this work, we propose a systematic approach to allow the agent to
make such a distinction for each action by tracking causal
relationships between goals within the current plan. This allow us to
determine an appropriate dispatch strategy for execution, based on the
nature of the goals this action contributes to. While our approach
manipulates the plan structure, it is worth mentioning that it is
focused on more on plan execution within the agent than in the process
plan synthesis.  For the purposes of this paper, it allows us to
consider planning in abstraction while providing an anticipatory
approach to continuous robotic plan execution.

The structure of this paper is as follows. We motivate with an
scenario illustrating the problem of flexible plan execution, discuss
previous efforts with robotic controllers for planning and dispatching
for execution. We then present the algorithmic detail to solutions
that allow decision at execution time when planned actions should be
started. Finally, after presenting experimental results, we conclude
and discuss potential directions for future research.

%%% Local Variables: 
%%% mode: latex
%%% TeX-master: "aaai13"
%%% End: 

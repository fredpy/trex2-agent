\section{Contribution}

However, an issue with finding a balanced approach is deciding when to
use least-committed and when to us EST. There is a clear difference
between the two approaches but when should, for example, the shopping
agent wait or start early. Lets demonstrate an ideal scenario where
the shopping agent balances both approaches. The agent, waking up at 8
AM, wants to buy some pants and t-shirts and needs to be home by 8
PM. The agent then leaves as early as possible to start shopping. The
agent continues to walk around and shop throughout the day because
there is no reason to rush home yet. Around 7 PM, the agent decides
that its time to go home so that it can make it there by 8 PM. What we
have demonstrated is a clear example of how the shopping agent can
start early while procrastinate about going home. This gives the agent
some flexibility to do more shopping in the day, if needed, while
still accomplishing its goal of being home by 8 PM.

Now lets look at how to create this ideal agent. There is clearly a
distinction being made between two different types of goals. There are
the wants of the agent, which is buying stuff, and the requirements
like being home by 8 PM.  The agent starts as early as possible when
it wants something, using EST, and waits until it has to for the
requirements, using least-committed. This distinction is a simple but
effective way of balancing the two approaches. In planning, the goals
can have sub-goals or conditions that must be met before the goal can
be completed. Lets say the Shopping agent wants an apple then a
sub-goal would be for the agent to buy an apple and even further to go
to a store that sells apples. All these sub-goals are linked to the
original goal and thus should be considered as goals
themselves. Depending on the original goal, the new goal will either
follow the ETS or least-committed approach.

However, we need to define a relation between the goals and their
sub-goals, which will allow us to easily identify all the
sub-goals. As stated earlier, the sub-goals are conditions that must
be met before the original goal can be achieved. The conditions are
not linked directly to the original goal, the sub-goals are actually
the conditions to an action that will satisfy the original goal.

\documentclass[letterpaper]{article}
\usepackage{aaai}
\usepackage{aaai}
\usepackage{times}
\usepackage{helvet}
\usepackage{courier}
\usepackage{graphicx}
\usepackage{algorithm}
\usepackage{algpseudocode}
\usepackage{amsmath}

% comment out this line to disable inline comments
\def\draft{}
\linespread{0.98}

\usepackage{color}
\usepackage{amsmath, amsfonts, amssymb}

\usepackage{subfig}

% \def\rx{{\texttt{T-REX\ }}}
% \def\rxe{{\texttt{T-REX}}}
\def\rx{{\texttt{FX\ }}}
\def\rxe{{\texttt{FX}}}
\def\eut{{\texttt{EUROPA}$_2$\ }}
\def\eu{{\texttt{EUROPA}\ }}
\def\eue{{\texttt{EUROPA}}}
\def\eus{{\texttt{EUROPA}'s\ }}
\def\nd{{\texttt{NDDL\ }}}
\def\nde{{\texttt{NDDL}}}
\def\mb{{\texttt{MBFD\ }}}
\def\mbe{{\texttt{MBFD}}}
\def\WW{{\texttt{WireWalker\ }}}
\def\WWe{{\texttt{WireWalker}}}

\def\etal{{et al.\/}}
\def\eg{e.g., }
\def\ie{{i.e.,\ }}
\def\etc{{etc.\ }}
\def\situ{{in situ \/}}
\def\PN{{\emph{PN} }}
\def\can{{\texttt{CANON\ }}}
\def\od{{\texttt{ODSS\ }}}
\def\ode{{\texttt{ODSS}}}

\newtheorem{Prop}{Proposition}
\newtheorem{Theorem}{Theorem}
\newtheorem{Lemma}{Lemma}
\newtheorem{Corrolary}{Corollary}
\newtheorem{definition}{Definition}

%\setlength{\marginparwidth}{8mm}

\usepackage{xcolor}

\ifdefined\draft 
  \usepackage{pdfcomment}

  \newcommand{\comment}[2]{\pdfcomment[#1]{#2}}
  \newcommand{\highlight}[2]{\pdfmarkupcomment[#1]{#2}{}}
\else
  \newcommand{\comment}[2]{}
  \newcommand{\highlight}[2]{#2}
\fi

\newcommand{\kcomment}[1]{\comment{author=Kanna,color=red}{KR: #1}}
\newcommand{\fcomment}[1]{\comment{author=Frederic,color=gray}{FP: #1}}
\newcommand{\pcomment}[1]{\comment{author=Philip,color=yellow}{PC: #1}}


\newcommand{\khighlight}[1]{\highlight{author=Kanna,color=red}{#1}}
\newcommand{\fhighlight}[1]{\highlight{author=Frederic,color=gray}{#1}}
\newcommand{\phighlight}[1]{\comment{author=Philip,color=yellow}{#1}}



\begin{document}

% \title{Anticipatory Continuous Robotic Plan Execution}
\title{Unanticipated Goals Inclusion Issue in Continuous Robotic Plan Execution}
%\author{Paper \# 88}
% \author{Philip Cooksey \And Fr\'ed\'eric Py \And Paul Morris \And Kanna Rajan}
\author{Philip Cooksey\\
\small{California State University}\\ 
\small{Monterey Bay}\\
\small{Seaside, California, 93955}\\
\small{\emph{pcooksey@csumb.edu}}\\
\And Fr\'ed\'eric Py\\
\small{MBARI}\\
\small{7700 Sandholdt Road,}\\
\small{Moss Landing, California, 95039}\\
\small{\emph{fpy@mbari.org}}\\
\And Paul Morris\\
\small{NASA Ames Research Center}\\
\small{Moffett Field, California, 94035}\\
\small{\emph{Paul.H.Morris@nasa.gov}}\\
\And Kanna Rajan\\
\small{MBARI}\\
\small{7700 Sandholdt Road,}\\
\small{Moss Landing, California, 95039}\\
\small{\emph{Kanna.Rajan@mbari.org}}\\
}

\maketitle{}

\begin{abstract}

  Robotic plan execution has traditionally assumed that goals are
  articulated prior to mission execution. As robots have become
  persistent and increasingly moved into real-world environments, this
  assumption is not necessarily true; for instance a user can decide
  to give a new objective to the robot for inclusion in the plan being
  formulated, add newer goals or modified others queued for
  execution. In most systems this leads, at best, to suboptimal final
  plan trace and possibly to the exclusion of objectives. Either of
  which could have been avoided should the robot have executed 
  its initial plan differently.
 % As robots have become
 %  persistent and moved into the real-world, the need to anticipate and
 %  capture evolving needs with new or reformulated goals becomes
 %  critical. 
  % We explore this issue not yet described in the literature.
  We first articulate and then demonstrate an initial approach to this
  problem motivated by a marine robotics domain with an execution
  policy that is sufficient to disambiguate actions for execution
  within a flexible temporal continuous plan execution system.  The
  resulting algorithmic complexity is linear in the number of actions
  and causal links of an existing partial plan.

\end{abstract}

\section{Introduction}
\label{sec:intro}

As autonomous robots become more robust and persistent, the likelihood
of receiving new or modified objectives \emph{during} an ongoing
mission is high. Generative continuous planning, integrated within the
robot's architecture, is one approach to enhance persistence while
dealing with evolving objectives and unanticipated situations.
Earlier works,
\cite{AmbrosIngerson88,Haigh98,alami:1998p820,mus98,chien99,mus04,py10},
have contributed control architectures, which embed planning engines.
They support a rich representation dealing with durative actions,
while ensuring partial plans are flexible for robust execution
\cite{lemai-chenevier2004}.

% such as \texttt{IPEM} \cite{AmbrosIngerson88}, \texttt{ROGUE}
% \cite{Haigh98}, and the LAAS architecture \cite{alami:1998p820} have
% all contributed to this field of research. Results from such work have
% lead to goal directed architectures that have sucessfully worked in
% the real-world such as the Remote Agent Experiment \cite{mus98}, the
% Autonomous Spacecraft Experiment \cite{chien99} and more recently the
% Teleo Reactive Executive \cite{py10}. Often these architectures embed
% planning engines that support a rich representation dealing with
% durative actions and resources while ensuring partial plans are
% flexible for robust execution \cite{lemai04}.  % A resulting consequence
% has been results in efficient approaches to temporal plan dispatch to
% ensure robust and sound execution \cite{mus98a,morris01}. 

While these approaches have been necessary to correctly execute a
\emph{flexible temporal plan}, they view the agent itself as
synchronously fulfilling goals provided a priori. They do not take
into account the introduction of new goals which could alter the
mission state and execution during mission execution.  Consequently,
the natural choice was to make the agent execute its actions as early
as the temporal plan would allow, so that the the robot would finish
its plan as early as possible providing more ``temporal room'' for
potential change in the plan. Such an assumption has proved to be very
efficient in dealing with unanticipated execution threats with an
exception in dealing with dynamic controllability \cite{morris01} ---
acting as early as possible keeps the remaining partial plan
flexible. Moreover, should the plan break, the robot has more time
left to identify and execute an alternate recovery strategy. An early
dispatch policy has therefore proved to be the most efficient
approach for the robot to robustly execute its plan when the set of
goals for the mission are fully instantiated.

Our experience in marine robotics shows that in a number of instances,
the full set of objectives of the robot is not known a priori. While
our missions last on the order of a day, it is often feasible to have
an initial plan that could be executed in a few hours. For example, a
typical mission can start with the scientific goal of collecting data on
an area for half of the mission and an operational goal to be at the
recovery location by the end of the mission. During this time, a
scientist receives and analyzes data collected by the robot, which
allow him to produce new goal(s) for dispatch to the robot, while
still in the water.  While starting an action proactively can often
add flexibility to the current plan, it can also at times negatively
impact the outcome should a new goal directly conflict with an action
undergoing execution.

In this paper we focus on a study demonstrating that strict execution
policies can present problems when new goals can emerge at any time
within mission scope; we are agnostic to the planning process
itself. We provide a simple solution that exploits temporal plan
structure to identify the execution policy of the next potential set
of actions. % based on the goals they are contributing to
% motivated by our domain.



% Our work is situated between the planner and the executive. Therefore,
% we are working with an already generated plan and are helping to determine
% an execution policy for every action. 
% The novelty of our work is twofold. First, we allow the agent to make
% such a distinction \emph{during} execution, for each action, by
% tracking causal relationships between goals. Second, while the initial
% approach is simple yet elegant, it is motivated by our real-world
% domain in the context of a continuously planning and executing robotic
% explorer with focus on execution rather than plan synthesis.

% This
% allows us to determine an appropriate dispatch strategy for execution,
% based on the nature of the goals this action contributes to. While our
% approach manipulates the plan structure, it focuses on plan execution
% within the agent rather than plan synthesis.  For the purposes of this
% paper, we consider planning in abstraction while providing an
% anticipatory approach to continuous robotic plan execution.

% The structure of this paper is as follows. We motivate with a
% scenario illustrating the problem of flexible plan execution, discuss
% previous efforts with robotic controllers for planning and dispatching
% for execution. We then present the algorithmic detail to enable the
% agent to decide at execution time when previously planned actions 
% should be started. Finally, after presenting experimental results, 
% we conclude and discuss potential directions for future research.

%%% Local Variables: 
%%% mode: latex
%%% TeX-master: "aaai13"
%%% End: 


\paragraph{An Illustrative Example} 

Our focus of research is in the oceanographic domain with a robotic
autonomous underwater vehicle (AUV), which illustrates issues arising
with dynamic goal emergence and motivates the approach in this paper.
In this domain, an AUV can either transit from one location to another
within the water column, or survey a location in order to collect data
about a feature of interest; an example being a hydrothermal vent
depicted in Fig. \ref{fig:Example}.  Note that the AUV must pass
through {\em Vent1} to get to {\em Vent2}.  The vehicle is deployed
initially with the scientific objective to sample {\em Vent 2} and the
operational goal to be back at the {\em Surface} by the end of the
mission which lasts $12$ hours, limited by its energy carrying
capacity. The support ship can send short commands to the AUV and
receive limited data via limited acoustic communications.

%%%%%%%%%%%%%%%%%%
% Use the following if really needed
% \subfloat[\small Bathymetry of vent sites off of NW United States]{\label{fig:ex:axial}\includegraphics[scale=0.35]{figs/vents.pdf}}
%%%%%%%%%%%%%%%%%%
\begin{figure}[!t]
  \centering
  \subfloat[]{\label{fig:ex:graph}\includegraphics[width=0.45\columnwidth]{figs/auv_example}}
  \hfill \subfloat[] {\label{fig:ex:plan}\includegraphics[width=0.55\columnwidth]{figs/example_plan}}
  %\hfill \subfloat[\small Initial planning problem.]{\label{fig:ex:init}\includegraphics[width=0.49\columnwidth]{figs/example_initial}}
  \caption{\small{(\ref{fig:ex:graph}) illustrates the domain and
      (\ref{fig:ex:plan}) an initial plan for the problem. The AUV is
      initially at {\em Surface} at 8:00 and the mission is expected
      to complete by 20:00 with the goal --- marked with thick borders
      --- to sample {\em Vent 2} and the operational goal to be back
      at the surface for recovery by mission end.}}
  \label{fig:Example}
  \vskip-5mm
\end{figure}

Given the initial problem in Fig. \ref{fig:ex:graph}, the onboard
planner produces the flexible temporal plan shown in
Fig. \ref{fig:ex:plan} for execution. However, at any time during the
mission scientists can decide, for example, that they also want to
{\em sample} {\em Vent 1} as a consequence of data processing during
the traverse from the surface to {\em Vent 2}.  Time on site for such
bathymetric exploration is limited and expensive, so dynamic goal
generation is a reality.  Finding a balance between executing a plan
action \emph{proactive}ly or waiting is the dichotomy that the plan
executive needs to resolve failing which it impacts the efficiency of
the mission. 

% . Otherwise, the plan will suffer along with
% the efficiency of the mission.

% This raises the issue of balancing the decisions during plan execution
% between staring one plan action as early as possible --- which we will
% call {\em proactive} -- or wait until the action should necessarily
% start -- called later. There is a clear difference between the two
% approaches but when should, for example, the AUV wait or start early
% and how either will impact the mission execution when the new
% objective will be integrated by the AUV ?

%\begin{figure}[!htb]
%  \centering
%  % \fcomment{In all the figures goals' color are the same "grey" as the
%  %   other actions .... need to correct this.}
%  \includegraphics[width=0.8\columnwidth]{figs/example_plan}
%  \vskip-2mm
%  \caption{\small The flexible plan solution of the AUV domain in
%    Fig. \ref{fig:Example}}
%  \label{fig:ex:plan}
%\end{figure}

Exclusively {\em deferring} the actions is not acceptable since it
results in the vehicle sitting at the surface, diving only at the very
last minute. Subsequently, if a new goal arrives there is no time left
to do both objectives leaving the robot to exclude one of the goals or
fail to be at the surface as planned by 20:00. Equally, the {\em
  proactive} approach can be problematic, since the vehicle will go
through the actions of the plan and be back at the surface early
(12:21). Even if the new goal is provided early enough (by 14:00), the
vehicle will have to dive back in order to survey and sample {\em
  Vent 1}.
% Even though the plan did initial provide a solution optimal in term of
% makespan the resulting mission scenario is not anymore as the vehicle
% went back and forth during the mission. Moreover in this specific case
Since most vents are usually deep ($1500$ meters or deeper often
taking $3$ or more hours to dive to depth) such a situation results on
two extraneous action, of at least an hour each, that could have been
avoided should the AUV have stayed at the bottom.  Neither of these
approaches are satisfactory; {\em defer}ment can lead to inefficiency,
while both deferment and \emph{proactive} execution can lead to
insensitivity to additional goals.
% Neither of these approaches are satisfactory; {\em defer}ment can lead to
% sensitivity to additional goals, while \emph{proactive} execution
% can lead to inefficiency.
Fig. \ref{fig:ex:proactive}, is an example of a solution aimed at
\emph{proactive} execution.

% in the plan execution not
% being robust to the addition of new goals while blind {\em proactive}
% execution can result on redundant actions within the mission that
% negatively impact the ability of the robot to do as much as it
% could.

% the executive to alter
% between the two divergent policies during the mission.

\begin{figure}
  \centering
  \includegraphics[width=0.65\columnwidth]{figs/example_early}
  \vskip-3mm
  \caption{\small A proactive solution for the plan from
    Fig. \ref{fig:ex:plan} resulting in a Surfacing event at the end
    of mission lasting in excess of $7$ hours.}
  \label{fig:ex:proactive}
  \vskip-5mm
\end{figure}

After analyzing the objectives, we note that these policies are
because of divergent goal semantics. While visiting {\em Vent 1} was
reflecting a \emph{scientific need}, the goal to be back at the
surface is an \emph{operational requirement} to avoid risk due to the
battery running out. While fulfilling both goals are equally important
for mission success, they do not carry the same notion of urgency. It
is desirable for the AUV to visit vents as early as possible but the
motivation to return to the surface is at the end of the
day. Therefore, a more balanced policy would be to alternate between
the two different approaches depending on what objective(s) the next
action contributes to. The AUV could have been {\em proactive} until
it starts to survey {\em Vent 2} and then {\em defer} heading back to
the surface, continuing to survey the same area, until it either
receives a new goal or the latest start time of the {\em Go} to
\emph{Vent1} action is reached ($\sim$ 17:50). In the event of a new
goal,
% or it is
% really time to go back to the {\em Surface} (around 18:00). Therefore
% when the AUV receive the new objective to sample {\em Vent 1} it would
% be still underwater and will just need to alter its plan to go in this
% new location. 
the resulting mission scenario would then be more efficient in terms
of its \emph{makespan}. We use the latest start time which does not
compensate for uncertainty in the plan and can be easily swapped with
another policy.  Our approach revolves around marking mission
objectives with a policy depending on the exploration of the plan
structure, during plan execution, to identify if the action is linked
to a goal -- in which case it needs to be executed {\em proactively}
-- otherwise leading towards a {\em deferred} execution policy for the
action.
 

% We demonstrate a scenario where the AUV ideally uses the two
% approaches at different times. The initial problem is illustrated in Figure
% \ref{fig:Example}.

% The AUV mission starting at 8 AM needs to {\em sample vent2} and
% return to the {\em ship} by 8 PM. The AUV starts traveling immediately
% to {\em vent1}. Considering that that it takes one to two hours to go
% from the {\em Surface} to {\em vent1}, roughly ten minutes to go from
% {\em vent1} to {\em vent2}, and more than two hours in order to {\em
% survey} and {\em sample}, a general plan solution is presented in
% Figure \ref{fig:ex:plan}. The plan presented here is partially
% instantiated giving the AUV the freedom to decide {\em when} to start
% each action within the valid boundary of the solution. For example,
% the AUV should go early in order to {\em sample vent2} so that the
% scientists have the a possibility for requesting more
% tasks. Conversely, heading back to the surface early would waste
% valuable time, roughly two hours, if the scientists decide they want to
% {\em survey} another location. Though by 6pm, it should go to the
% surface so the scientists can pick it up by 8pm. In this scenario, we
% see that the AUV alternated between deciding to execute actions early
% or {\em defer} them depending on the nature of the action it needed to
% take next, or more accurately the nature of the objectives related to
% this action. The AUV was {\em proactive} on traveling to {\em sample vent2}
% because the scientists want it to be completed. On the other hand, the
% AUV has to return to the surface by 8 PM, however, the scientists don't
% explicitly want this done, allowing it to procrastinate.  By
% doing so, the AUV is available to complete new tasks given to it by
% the scientist.\fcomment{One concept that would help if introduced is
%   around the same notion of task span in plan optimization : we have
%   the same problem here but we use a more refined execution policy in
%   order to avoid extraneous actions should a goal come back to us.}



% While this example may appear academic at first, it reflects situations
% we have seen within embedded agent execution in our domain. Indeed, we
% do daily operations wehere our AUV is deployed and scientists can
% remotely send new objectives as the mission goes along to the vehicle
% as they see new areas of interest. Especially in the upper water
% column the nature of the area to be examined depends highly on the
% dynamic of the ocean itself and is difficult to predict beforehand. Therefore,
% scientist can use data sampled by the AUV or other sources (eg
% satellite data, ship based observations, \dots{}) during the beginning of
% the operation in order to give it better informed objectives for the
% rest of the operation. At the same time the vehicle has
% also operational objectives such as going to a place where its
% recovery will be easier for operators. This gives a similar
% distinction between the science objectives and operation
% objectives. Similarly to our example we do not really want the
% AUV to get back to recovery area too early as a new science goal could
% be sent to it which in turn would rather be fulfilled as early as
% possible.

% This paper discusses the problem of dispatching when trying to execute
% a plan. In particular, dispatching in a dynamic environment where the
% plan is expected to change due either to unanticipated events or external
% requests with new directives. External requests  can occur
% at any time which make them in essence uncontrollable events.
% Specifically, we focus on how these new requests, coming from the
% external world, alter the way we need to dispatch the plan, rather than
% how they will be integrated into the plan or any part of the planning
% process. The reason for our separation from planning is that
% oftentimes planning and executing are split up into two different
% jobs. Often times a robotic agent is given an already created plan, and it
% must then choose when to execute parts of the plan. Therefore, our focus is on
% developing a method for dispatch a plan, after it has already been created, while
% understanding that new requests may come in the future.

% The approach we have taken on dispatching looks at the token level of a plan,
% specifically at tokens generated from external requests which we define as goals.
% Because they have been requested by an external person with the intent of being 
% completed promptly, they have a high priority. In contrast, there are tokens that only describe the
% evolution of a timeline, which we define as non-goals. In order to keep the plan 
% valid, the agent is obligated to complete the non-goals, but there is no rush. Thus, the
% non-goals have a low priority. Therefore, we want to complete the goals
% as early as possible in order to give adequate time for the possibility of new
% goals, and complete the non-goals as they become necessary for the validity of the plan. 
% Some may argue that finishing the goals early doesn't guarantee that
% there will be enough time for new goals, however, that is an issue with planning, 
% and our concern is whether dispatching caused the waste of time.


%%% Local Variables: 
%%% mode: latex
%%% TeX-master: "aaai13"
%%% End: 


\section{Previous Approaches to Dispatching}

% \fcomment{I need to rework this as it is not really a state
%   of the art in the current form ... we also need to present that fact
%   that not that much people do consider postponement of actions
%   ... except maybe for dynamic controllability problem ... this may be
%   the key for this part. WE can present proactive approaches as being
%   the general agreement except for the special case of dynamic
%   controllability where 3DC+  introduces wait actions which in fact put
%   time-points that need to be deferred. We on the other hand try to
%   look at another aspect that may trigger the decision to defer an
%   actions based on its purpose}

Dealing with plan execution is not a new problem and we can find a lot
of work that relates to this during the last decades. Still, it is
pretty rare to see work that envisage that actions could be
postponed except when this is necessary to not break the current
plan. 


The most prominent work is related to the dispatchability of simple 
temporal networks (STN) \cite{mus98a}. The core of the problem is to
ensure that the temporal constraints can be propagated efficiently
within the plan in order to allow the executive to decide quickly
whether an action should be started or not while ensuring the plan
consistency. In order to accomplish such a task, the STN supporting
the plan to be dispatched is transformed into a All-Pairs network and
stripped of unnecessary edges, resulting often on a more compact
temporal network and lessen the propagation cost of its updates. The
role of the executive is to select time-points within the current
execution bounds and propagate its value within the simplified
network. Still, while this work contribute to ensure that execution
time are correctly propagated within the plan with a limited cost, it
does not directly address how to decide what value should be set for a
given time-point in the scope of its possible values. More specifically
it is still the role of the executive to decide whether it should
start an action as early as possible or consider it as not urgent. 

When dealing with least-commitment planning solution this decision is
deferred to the plan executive. For example in \cite{mus98c}, the 
executive is defined as having two responsibilities: the selection and scheduling of plan
events for execution. The executive needs to
be highly reactive as it is necessary to function in a real-time
environment. One solution offered for dispatching events efficiently
is the proactive approach. This approach greatly reduces the plan
flexibility, and therefore robustness, as all start time-points are
grounded to a specific value which is compatible with the initial
constraints. In order to avoid having a procrastinating system the
overall agreement in term of policy is to start an action as early as
possible (ref ?). 

% \fcomment{Therefore, one reoccurring problem of dispatchability is finding a way
% of balancing efficiency, proactive, with flexibility,deferred.}
\fcomment{We need to update this to reflect that it is not ETS}
There have been many uses of these two approaches when dispatching plans, as
previously shown, but a compromise to one is most often used rather than
a balance. Demonstrated in the tool MAPGEN \cite{bresina03} which used
the Earliest Time Solution (ETS) for generating and displaying plans
very quickly. Then allowing the users to manipulate the plan afterwards,
getting around the problem of ETS generating undesirable plans. What is
needed is a way of using both proactive and deferred together giving a
balanced approach to dispatching.

While in the general case it can be acceptable it
may become problematic when put in he context of potential new
objectives emerging during the mission.  Take our shopping example,
and apply the proactive approach globally -- assuming that all actions
can be completed on their minimum time.  As shown in 
Fig. \ref{fig:ex:proactive}, the proactive approach allows the
agent to get its apple safely before noon but as we go along with the
remaining part of the plan it results in the agent getting back home
by 9:05 and being stuck in the context of the current plan for the
next 10 hours and 55 minutes. Should he receive a call from his wife
to buy extra things he will then be forced to re-plan accordingly and
get back and forth between his home and the stores all other
again. By being blindly proactive the agent made its overall strategy
less efficient than if he took the option to procrastinate at the mall
until he needs to get back home. 


\begin{figure}
  \centering
  \includegraphics[width=0.8\columnwidth]{figs/example_early}
  \caption{proactive solution for the plan from Fig. \ref{fig:ex:plan}}
  \label{fig:ex:proactive}
\end{figure}

One case we have seen in literature where a time-point is
considered to be deferred is related to the dynamic controllability
issue with STNs with uncertainty (STNU). In \cite{morris01}, they propose 
an algorithm that during planning insert wait constraints within the
solution temporal network that will allow one time-point to wait until
an uncontrollable event occurs. In this case, the deterrence of the
execution of this time-point is enforced in order to ensure robust plan
execution despite exogenous uncontrollable events. In
\cite{gallien2006},  this approach is further discussed along with the
Makespan issue while dealing with least-commitment planners which can
insert unjustified waits within the plan potentially decreasing in turn the
overall performance of the system. Both of these approaches show that
when dealing with uncontrollable temporal constraints -- such as for
example the duration of a navigation task which depends on external
factors -- it is necessary to defer some actions in order to ensure
the plan execution. 

An alternative approach is related to wrok with soft constraints
within the temporal domain indicating preferences on actions timing
such as in \cite{khatib2001temporal}. This should provide  a general
solution for users to specify whether actions of the plan should be
preferably started. One problem with general frameworks dealing with
soft constraints though is their poor performances \cite{bartak2002}
as the general problem is NP-hard. Even though it can be reduced to a
more tractable (polynomial) solution by specifying preferences as
convex functions \cite{rossi2006learning}, the  
complexity is still at on the cubic in relation to the number of
variables. Moreover the solutions proposed do optimize the plan {\em
  a-priori} reducing then the plan flexibility which in turn make it
in  turn more prone to fail during execution.

Our work on the other hand is more complementary to previous work, indeed
while we do not really deal with dynamic controllability in our
problem -- which can anyway be treated as in this previous work -- the
core of our problem is not that much the observation of events during
execution (which is more a observation related impact) but instead we
are more focused on the knowledge that, within our agent, new
objectives can emerge at any time, and we want to avoid as much as
possible doing actions which are not ``urgent''. similarly while we do
not really deal with complex preferences specification that can help
improve the planned solution our purpose is to defer the decision to
start to execute actions within the plan to the executive allowing to
quickly adapt the plan without heavily relying on plan-repair or
re-planning solutions that could badly impact the system reactiveness.
By doing so, we take the risk of doing unnecessary actions as new
goals need to be integrated in the plan.

% s
% considered to be deferred is related to the approach proposed in 
% \cite{morris01} to deal with dynamic controllability issue. 
% \fcomment{following will be included in former para. Also we need to
%   address that it is in fact outside of our scope here ... we actually
% do not address dynamic controllability, worse we tend to make waiting
% actions more brittle in general as we take the upper bound  for needed
% actions}

% The only work where we see 

% % Dispatchability for a simple temporal network (STN) has been identified
% % with whether a plan is capable of fast execution, retains its
% % flexibility despite uncertainty in the plan, and all information is
% % available at execution \cite{mus98}.  To accomplish the task of
% % efficient execution, the STN's are transformed into All-Pairs networks
% % and striped of their unnecessary edges, which will often create a more
% % compact temporal network and esse the propagation of its updates.
% % The executive then selects a timepoint, within the bounds of the 
% % execution time, and every timepoint before it has
% % already been executed. It then sets the start time to the selected
% % timepoint and propagates the needed changes throughout the graph
% % \cite{mus98}. The process of selecting a timepoint can lead to
% % issues when two timepoints have the possibility of being executed at the
% % same time, where one should necessarily come first, but the incorrect
% % one gets selected first. This mistake can cause the STN to become
% % brittle, which leads to a higher chance of failure in the plan.

% % \fcomment{Need to continue to refine from here and talk about STNU and
% %   morris01 as  the only approach we know of where postponed events are present
% %   though the inserted wait}

% % Take our shopping example\fcomment{figure again}.  We'll say the agent wakes up
% % at 9 AM and the duration for traveling to the store is forty minutes
% % and buying the apples is ten minutes. Using the proactive approach, the
% % agent will start shopping as soon as possible and thus be done shopping
% % by 10 AM. The agent will then go directly home afterwards to satisfy the
% % ``home by 8 PM'' goal. However by doing so, the agent will be back
% % home by 11am with the constraint to stay at home for the next 9
% % hours. 
% % \fhighlight{If his wife} is then calling to ask him to buy extra things it will
% % results on hime to need to fully relax his current plan and go back
% % and forth between its home and the store all other again. By being
% % fully proactive the agent.\fcomment{Maybe what was meant initially
% % here is more related to dynamic controllability which mean that I need
% % to twek this text to be more related to a modelled
% % but uncontrolbale situation.}


% % However, the start time for this goal will be roughly
% % 11 AM, and if the agent wants to get some dessert before dinner it will
% % cause a failure in the plan. What this example shows is how EST can
% % severely reduce the flexibility of a mission. EST can also cause
% % inconsistency within the plan.

%  {\em \fhighlight{Therefore,} \cite{mus97}\fcomment{See comment before} added
% implied links between timepoints allowing the planner to find them
% efficiently.\fcomment{I don't get what you mean here}}


% Another issue with execution is uncertainty in the plan. The above
% example illustrates the uncertainty about when the agent should go home
% or if the agent will want to buy more food during the mission. A
% solution for execution uncertainty is the least-committed method, which
% waits until that last moment to execute events. This approach keeps the
% plans flexibility and it allows the executive to have all of the
% available information before making a choice (Block, 2006). However, it
% has been argued that the least-committed method is an inefficient
% solution because in a real-world environment an agent might idle for
% long periods of time (Gallien, 2006). This can severely decrease the
% overall performance of the system. Lets demonstrate the least-committed
% policy in our Shopping agent example using the same goals and durations
% as before. This time around the agent won't want to do anything until it
% is absolutely necessary to do it. The agent will then wait until 6:20 PM
% before leaving and will finish shopping at 7:20 PM this way the agent
% can be home by 8 PM. However, even though this keeps flexibility in the
% plan there are a few issues. If there happens to be traffic then the
% agent might not make it home by eight PM. Also the agent procrastinates
% all day just waiting to go shopping rather than starting early on in the
% day, which wastes valuable time. Their (Gallien, 2006) solution is to
% change the planning heuristic altering how their plan is created
% inevitably adjusting it for faster execution.


%%% Local Variables: 
%%% mode: latex
%%% TeX-master: "aaai13"
%%% End: 

papr
 
\section{Planning Definitions}
\label{sec: defs}

The focus of this paper is on the possibility accommodating \emph{new
  objectives} arising during mission execution, with consequent need
to preserve context for their inclusion and subsequent execution and
that actions not deemed 'urgent' be deferred. Our robot generates and
dispatches a partial plan in continuum to a lower level controller
which further decomposes these plans towards atomic actions. These
actions ultimately actuate hardware on our AUV. The world responds by
sending sensory feedback as \emph{observations} up thru the
hierarchical controller \cite{mcgann08bdup,rajan12dup}. Details of the
controller are beyond the scope of this paper and are omitted without
loss of generality of the concepts described. We first consider some
basic representational definitions related to temporal planning
consistent with \cite{Nau:2004}:

\begin{definition}
  \label{def:domain}
  A temporal planning domain is a triple $D = ( \Lambda_\Phi, O, X )$, where:
  \begin{itemize}
  \item $\Lambda_\Phi$ is the set of all temporal databases that can
    be defined with constraint, constant, variable, and relation
    symbols in the representation.
  \item $O$ is a set of temporal planning operators.
  \item $X$ s a set of domain axioms.
  \end{itemize}
\end{definition}

A \emph{token} is a property that holds within a flexible temporal
interval which represents an action or a state associated with $O$ or
$X$ \cite{py10dup}. A temporal interval is delimited by two {\em
  timepoints} \cite{Boddy93} marking the possible values of its {\em
  start} and {\em end}.

\begin{definition}
  \label{def:problem}
  A temporal planning problem in $D$ is a tuple $P = ( D, \Phi_{obs}, \Phi_g )$, where:
  \begin{itemize}
  \item $\Phi_{obs} = (F, C)$ is a database in $\Lambda_\Phi$ that
    satisfies the axioms of $X$.  $\Phi_{obs}$ represents an initial
    scenario that describes initial state of the domain and
    observations.
  \item $\Phi_g = (G, C_g)$ is a database that represents the goals of
    the problem as a set $G$ of tokens together with a set $C_g$ of
    objects  and temporal constraints   on variables of $G$.
  \end{itemize}
\end{definition}

\begin{definition}
  \label{def:plan}
  A plan is a set $\pi = \{a_1,...,a_k\}$ of actions, each being a
  partial instance of some operator in $O$.  We define $\lambda$ as
  the state-transition function.  $\pi$ is a solution of a problem $P
  = (D, \Phi_{obs}, \Phi_g)$ iff there is a database in $\lambda(\Phi_{obs},
  \pi)$ that entails $\Phi_g$.
\end{definition}
 
While the focus is on executing plan $\pi$, to reflect the dynamic
interaction of the agent with the environment we need to refine the
definition of the sets $\Phi_{obs}$ and $\Phi_g$. As the world
evolves, new (or refinement of existing) observations are added to
$\Phi_{obs}$. Similarly, the agent operator can request new goals in
the future to be added to the agent's $\Phi_g$ as mission time
advances. For the sake of simplicity we consider that the alteration
of these sets is purely additive with time\footnote{In practice, their
  evolution is more complex as completed goals becomes observation;
  even a goal previously requested can be cancelled by the
  operator. While this assumption largely simplifies problem
  description, our algorithm works regardless.}:
\[ \forall \{t, t'\}: t \le t' \Rightarrow \Phi_{obs}(t) \subseteq \Phi_{obs}(t')
\wedge \Phi_g(t) \subseteq \Phi_g(t') \] 

where $\Phi_{obs}(t)$ and $\Phi_g(t)$ are the value of these sets at
time $t$.

$\Phi_{obs}$ grows dynamically, reflecting cumulative observations as
$\pi$ is executed. In nominal situations, new elements of $\Phi_{obs}$
are refinements of the plan -- for example by asserting that a planned
command just started\footnote{This precludes situations where new
  observations invalidate the plan; this is out of the scope of this
  paper}. The agent can also receive new objectives at any point with
those added to $\Phi_g$. This assumption has an impact on how it is
preferable to handle plan execution. Indeed while deciding when to
start an action within the plan, one need to make sure that the
execution of this action will not limit the ability of the agent to
treat potential future emerging goals. In the light of it the agent
should at the best of its knowledge try to balance the impact of the
next available action as early as possible or prefer to delay it in
the eventuality new goals occur. In our example, it was making sense
to go to Vent2 early, but on the other hand going back to the surface
too soon would result in locking the AUV -- within its current plan --
at the Surface until $8$ pm. The solution providing the most
flexibility for the AUV was therefore for it to alternate between the
two policies depending on the action impact.

In order to help the AUV have a better knowledge on the nature of the
goals we do consider that each goal provide information on its
priority. In that purpose, we define that $\Phi_g$ is partitioned into
two sets:

\begin{itemize}

\item the \emph{internal} goals $\Phi_{gi}$ which represent goals the
  agent {\em needs} to maintain internally. These goals will be
  considered as objectives that are not of the higher priority and
  therefore their actions can be deferred during execution.

\item the \emph{external} goals $\Phi_{ge}$ which represents the goal
  received by the agent externally. As these goals are requested by
  the user, we consider them as to be of higher important -- ie the
  agent {\em wants} to execute them. Therefore, their execution should
  be preferrrably proactive.

\end{itemize}

At any point we need to evaluate an action within our plan $\pi$ we
consider that this plan is up to date and provide a solution of all 
the goals of both $\Phi_{gi}$ and $\Phi_{ge}$ that can reasonably be
done within the current mission scope.

\section{Algorithmic Description}

As a new action can be dispatched for execution, the executive needs
to evaluate how it relates to the goals of the plan. Intuitively if
this action was generated by (or contributes to) an internal goal of
$\Phi_{gi}$ it needs to be taken proactively, while otherwise we can
consider it as non-urgent. Therefore, when evaluating if the token
representing this action within the plan the executive needs to do a
forward search on the causal links related to this token to see if
they lead to an internal goal as implemented in Algorithm
\ref{DispatchToken}.

\begin{definition}
\label{def:subgoalLink}
A goal $g$ is causally linked to a state value $c_i$ if there is an
action such as $g$ is an effect of this action and $c_i$ is one of its
conditions\footnote{For metric temporal plans, Allen Algebra
  \cite{allen84} is the basis for determining relationships between
  actions and is more general than pre-conditions in classical
  planning.}. 
% A causal link is defined as linking a goal as an effect of an action
% whose conditions are needed in order to complete the goal and, thus, are
% subgoals. This link can be recursive as the conditions themselves may
% be the effect of an action causing a causal chain to build.
\end{definition}

As a result all the conditions of this action can be assumed to be
subgoals with similar priority as $g$. 

\begin{algorithm} [H]
  \caption{\small The function $DispatchToken$ finds if there is a
    goal in $\Phi_{ge}$ that is connected to the token, $t$, and, if
    so, dispatches the token.}
  \label{alg:dispatch}
\label{DispatchToken}
\begin{algorithmic}
\small 
\Function{DispatchToken}{Token $T$}
\State Boolean $Goal = \textsc{SearchForGoal}( T )$
\If{ $Goal$ $=$ True}
	\State \Return Dispatch $T$
\ElsIf{ $T$ start upper bound $ \leq $ current tick}
	\State \Return Dispatch $T$
\Else
	\State \Return Don't dispatch $T$
\EndIf
\EndFunction
\end{algorithmic}
\end{algorithm}

Algorithm \ref{alg:dispatch} is the central deciding  point for how a 
token should be dispatched.
It is called at every execution cycles and evaluates the next tokens to be
executed.  By finding out that a token is connected to a goal
 in $\Phi_{ge}$, we conclude that the token is a sub-goal,
and thus dispatch it immediately being proactive.  On the other hand,
if the token is not connected to a goal then we defer dispatching it
until necessary.  This demonstrates our distinction between how we
dispatch tokens, proactive or deferred.

\begin{algorithm} [H]
  \caption{\small The function $SearchForGoal$ does a Forward search
    looking for a token that is in the set $\Phi_{ge}$.}
\label{SearchForGoal}
\begin{algorithmic}
\small
\Function{SearchForGoal}{ Token $T$ }
\If{$T \in \Phi_{ge}$}
  \State \Return True
\Else 
  \State List $Fringe$ = \{ Actions that $T$ is Condition of \}
  \ForAll{Action(s), $A$, in $Fringe$ }
     \ForAll{Effects, $E$ of $A$}
       \If{$E \in \phi_{ge}$}
          \State \Return True
        \Else 
           \State List $L$ = \{ Action(s) that $E$ is Condition of \}
           \State $Fringe$ = $Fringe \cup L$ 
        \EndIf
      \EndFor
  \EndFor
\EndIf
\State \Return False
\EndFunction
\end{algorithmic}
\end{algorithm}

To inform this decision we use Algorithm \ref{SearchForGoal} which
search for a causally link between the token $T$ and token $G$ of
$\Phi_{ge}$ in a forward-search manner. During our search, 
if we find $G$ then we know that the original token is part of the 
solution for completing $G$.  As such, we want to be
proactive in ordeto utlimately complete $G$ as early as possible.
If we don't find $G$ then the token has no connection to an
external request. The token still needs to be eventually dispatched, 
however, there is no motivation to execute this action earlier than necessary.

This algorithm is equivalent to a breadth first search along the
planning structure starting from the action we need to
evaluate. Therefore, its complexity is $O(N+E)$ where $N$ is the
number of tokens within the plan and $E$ the number ofcausal links that relate
these tokens. 

%\subsection{Dynamic solution during planning}
%
%Searching for a goal as Algorithm \ref{SearchForGoal} does can be
%quite computational expensive particularly if there are many tokens
%that are continuously being dispatched. Completing a full search every
%time a token needs to be dispatched can severely slow down the
%execution process, which needs to remain quick to ensure proper
%execution. Therefore, our next algorithmic approach distributes the
%full search within the creation of the plan.  Resulting in spreading
%out the full cost of the search. In order to not repeatedly search the
%plan, we save the tokens that are connected to a goal in $\Phi_{ge}$
%found during the search. In this way, we acquire a list of tokens,
%$List_{goals}$, that should be dispatched early.
%
%An alternative solution is to embed the propagation of these value
%during the planning search. The algorithm uses the same dispatching
%method as algorithm \ref{DispatchToken}.  The difference is that
%rather than searching for the goal using algorithm
%\ref{SearchForGoal}, it only searches the list, $List_{goals}$, to see
%if the token is in it. The actually searching for the goals and
%causally connected tokens happens in algorithms \ref{NotifyActivated},
%\ref{NotifyMerged}.
%
%\begin{algorithm}[H]
%\caption{\small Saves goals as they are added to plan}
%\label{NotifyAdded}
%\begin{algorithmic}
%\Function{NotifyAdded}{ Token $T$ }
%\If{$T$ is a Goal in $\Phi_{ge}$}
%	\State Insert $T$ into $List_{goals}$
%\EndIf 
%\EndFunction
%\end{algorithmic}
%\end{algorithm}
%
%\begin{algorithm} [H]
%\caption{\small Removes the token after it is removed from the plan}
%\label{NotifyRemoved}
%\begin{algorithmic}
%\Function{NotifyRemoved}{ Token $T$ }
%	\State Remove $T$ from $List_{goals}$
%\EndFunction
%\end{algorithmic}
%\end{algorithm}
%
%\begin{algorithm}[H]
%\caption{\small Searches for tokens connected to goals}
%\label{NotifyActivated}
%\begin{algorithmic}
%\Function{NotifyActivated}{ Token $T$ }
%\If{$T$ is a goal in $\Phi_{ge}$ or $T$ is linked to a goal through one causal link}
%	\State Recursively search the reverse causal link and add the tokens into $List_{goals}$
%\EndIf
%\EndFunction
%\end{algorithmic}
%\end{algorithm}
%
%\begin{algorithm}[H]
%\caption{\small Deactivates token in plan}
%\label{NotifyDeactivated}
%\begin{algorithmic}
%\Function{NotifyDeactivated}{ Token $T$ }
%\If{$T$ is not a goal and not one causally linked to a goal}
%	\State Remove $T$ from $List_{goals}$
%\EndIf
%\EndFunction
%\end{algorithmic}
%\end{algorithm}
%
%\begin{algorithm}[H]
%\caption{\small Searches plan when tokens are merged}
%\label{NotifyMerged}
%\begin{algorithmic}
%\Function{NotifyMerged}{ Token $T$ }
%\If{$T$ is goal in $List_{goals}$}
%	\State Recursively search the reverse causal link of the active token merged with $T$ and add tokens to $List_{goals}$
%\ElsIf{The active token of $T$ is in $List_{goals}$}
%	\State Recursively search the reverse causal link of $T$ and add tokens to $List_{goals}$
%\EndIf
%\EndFunction
%\end{algorithmic}
%\end{algorithm}
%
%\begin{algorithm}[H]
%\caption{\small Removes token when split}
%\label{NotifySplit}
%\begin{algorithmic}
%\Function{NotifySplit}{ Token $T$ }
%\If{$T$ is not a goal or not one causally linked to a goal}
%	\State Remove $T$ from $List_{goals}$
%\EndIf
%\EndFunction
%\end{algorithmic}
%\end{algorithm}
%
%In order to distribute the search, we situate our algorithm within the
%planning search of the Europa Planner,\cite{frank2003}, which offers 
%callback functions for when a token in the plan is altered. The majority
%of the searching happens when tokens are either activated or 
%merged. For a token to merge with another token it has to be compatible with
%another token already in the plan. Splitting happens when they are
%no longer compatible. We have designed our algorithm around the
%Europa Planner, however, we believe that the general approach
%can work on any other planner. 
%
%Taking full advantage of the planning search, we
%use a backwards search from the goal following the reverse causal link
%to the connected tokens.  We fully search from the goals because we
%know that all the tokens connected through the causal link are
%sub-goals. By contrast, fully searching each token could be
%wasteful because there is no certainty that it will be linked to a
%goal and, therefore, could bring little value to our search. However,
%some tokens may get added to the plan or linked to a goal after we
%have already searched the goals. Therefore, for every token we do a
%local forward search of one causal link to verify if it is connected
%to a goal in our saved list.  If so, we do a full backwards search
%from the token since it has now proven to be valuable. After the plan
%has been searched, it is as easy as searching a list for a token to
%see if it should be dispatched early or be deferred to later.
%
%This approach is potentially more costly than the previous algorithm 
%as it needs to do local updates whenever the plan is altered by the search
%including retracting past updates if a backtrack occur during the
%search. Still it is compelling in the fact that this cost occurs during
%planning reducing the decision problem during execution to simply
%check if the given action has been marked during planning. It has
%been the solution we have preferred within our system for this reason
%as it is functionally equivalent to previous algorithm while reducing
%extra computation cost as the plan is executed. Planning phases occur
%within the plan only when either the initial plan failed to execute or
%the set of goals has been altered. Therefore, it is safe to assume that
%planning should occur more sporadically than execution decisions 
%which then give an edge to this latter algorithmic solution. 

%%% Local Variables: 
%%% mode: latex
%%% TeX-master: "aaai13"
%%% End: 



\section{Experimental results}
\label{sec:exp}

{\em\color{gray} Need to present here both practical results that illustrates the outcome of our solution and how it benefits ... eg show a case where new goals are introduced as we go along}

{\em\color{gray} Also need some analysis -- potentially numerical -- on the overhead and impact fo both solutions relative to each other but also potentially to a more direct classic approach. 
To finally discuss why one solution was preferred on our system} 

Our experiment follows that of the original plan given in
Fig. \ref{fig:ex:plan}. At the beginning, the AUV need to {\em Sample Vent2}
and it has to return to the surface by the end of the mission. We
implemented this on our executive with plans being produced by the
europa planning engine \cite{frank2003}.\fcomment{I think it is
  important to give at least dome indication of the kind of tools used.
  this should be enough as it is not really giving away that we used
  trex}. Figure \ref{fig:ex:mixed1}
demonstrates the resulting plan from our algorithm and will guide our
explanation. Both algorithms will result in the same execution of the
plan but their approach is quite different.

\begin{figure}
  \centering
  \includegraphics[width=0.8\columnwidth]{figs/example_MixedInitial}
  \caption{\small The algorithmic solution for the plan from
    Fig. \ref{fig:ex:plan}. Solid lines indicate conditions and
    dashed lines indicate effects. Pentagons indicate {\em urgent}
    goals, stars indicate tokens that were deduced as {\em proactive}.}
  \label{fig:ex:mixed1}
\end{figure}

For Algorithm \ref{DispatchToken}, the search is quite straight
forward. For example in Fig. \ref{fig:ex:mixed1}, if we are {\em At
Surface} then the next {\em Go} token will be dispatched. Because the
search will follow the causal links forward, where upon, it will find
the goal {\em Sample Vent2} resulting in dispatching proactively.
Similarly, this happens for all of the tokens that are starred. The
next token to {\em Go} to {\em Vent1} is continuously searched but deferred for later dispatching,
because it is not connected to a goal. The resulting
AUV stays at {\em Vent2} rather than heading to the {\em Surface}
immediately like in Fig. \ref{fig:ex:proactive}.

For the distributed algorithm approach, each token is checked during
the creation of the plan to see if it is an external goal in
$\Phi_{ge}$, or connected to one through a causal link. When the {\em
Sample Vent2} is checked, we immediately find that it is a goal. We then
follow the reverse causal link and find {\em Survey Vent2}.  However to
better illustrate the algorithm, we can imagine that only the {\em Survey
Vent2} has been causally connected to the goal so far. Therefore, we
only star those two tokens. The path from {\em Surface} to {\em Vent1}
to {\em Vent2} has yet to be built. When the path has finally been
built, and {\em At Vent2} is checked, our algorithm searches one
causal link and finds {\em Survey Vent2} which is starred. The search
then follows the reverse causal link and stars the rest of the
path. The starred tokens will then be proactively dispatched while the
non-starred tokens will be deferred until later. Having similar
results to Algorithm \ref{DispatchToken}.

\begin{figure}
  \centering
  \includegraphics[width=0.8\columnwidth]{figs/example_MixedUpdate}
  \caption{\small The solution after receiving external request for
    the plan from Fig. \ref{fig:ex:mixed1}} 
  \label{fig:ex:mixed2}
\end{figure}

Our reasoning for keeping the AUV at {\em Vent2} is that
nothing is requiring it to go {\em Surface} as soon as possible and that
more external requests may come in the near future. To demonstrate
this imagine that while the AUV is at {\em Vent2} it gets a
request at 11:30 asking it to {\em Sample Vent1}.  We show the resulting plan in
Fig. \ref{fig:ex:mixed2}. Again, both algorithms will result in the
same conclusion. Algorithm \ref{DispatchToken} will search from {\em
Go(Vent2, Vent1)} and will now find {\em Sample Vent1}. The
distributed algorithm will find and star the new tokens when the plan
gets updated. The resulting new starred tokens will be proactively
dispatched.

As the end of the day approaches, the AUV will need to start heading
to the surface. At 19:00 both algorithms will find that the {\em Go(Vent1,
Surface)} is still not connected to a goal but that the upper bound time
for starting the token has been reached. Therefore, we will dispatch
the token because it has become necessary for completing the plan.
After returning to the {\em Surface}, the plan shows that the AUV will then {\em
Go(Surface, ?)} (dashed in the figures). This is an artifact resulting
from the plan model which specifies that a {\em At} token is followed by a {\em Go}. 
However, our algorithm will not be dispatching this token as 
it is not connected to a goal and its upper bound start time
($+\infty$) will not be met.

\begin{figure}
\centering
\includegraphics[width=\columnwidth]{figs/example_run.pdf}
\caption{\small  One mission run that shows the time needed for dispatching. Star represents when the new goal was
received. } 
  \label{fig:example_run}
\end{figure}

In Figure \ref{fig:example_run}, we show a simulated mission run that is similar to our original example in Fig. \ref{fig:ex:plan}.
However, in order to increase the complexity of the plan we added a total of ten locations, eight before the vents. The first goal 
is to sample {\em vent2} at location ten and the second is to sample {\em vent1} at location nine. The spikes in the time 
happen when it dispatches both an {\em At} and a {\em Go} which means that the graph must be searched twice. There
is a clear trend that as the AUV gets closer to the goal the time decreases. However, there is some variability in the time which
can most likely be accounted for as an error in the clock since the times being measured are less than one milliseconds. 
After the new goal is received at tick 127, the {\em Go} is no longer deferred gets dispatched. The search for Goal\#2 becomes 
relatively simple as it is not very far away in the plan, and thus the time decreases significantly.

\begin{figure}
\centering
\includegraphics[width=\columnwidth]{figs/HistogramAlg1}
\caption{\small  We ran 500 missions were Algorithm \ref{DispatchToken} was timed for every tick. Shows the
   distribution of the time.} 
  \label{fig:histogram}
\end{figure}

In order to demonstrate the average amount of time that Algorithm \ref{DispatchToken} uses, we ran 500 simulated missions
and timed every tick of the mission. Figure \ref{fig:histogram} shows that for about 90\% of the time, Algorithm \ref{DispatchToken}
takes less than .2 milliseconds to complete it's search. Less than 5\% of the time Algorithm \ref{DispatchToken} went above
two milliseconds with an upper bound time of thirteen.  





%%% Local Variables: 
%%% mode: latex
%%% TeX-master: "aaai13"
%%% End: 


\section{Discussions and future work}
\label{sec:conclude}
%{\em\color{gray} Need pargarph that resume what was presented}

Increased robustness in hardware has led to persistence of robots
often in hostile environments such as our oceans. With onboard
autonomy increased versatility has resulted in the need to anticipate
and adapt to goals during mission execution.While most of existing
approaches allow the emebedded planner to accept new goals during
mission execution, none of them provided a balanced approch to deal
with the tension produced by the potential emergence of these new
objectives. So far the only solution was to integrate such knowledge
within the domain description itself which results on a more complex
model. Our experience demonstrated that such added complexity 
makes the planning domain harder to maintin and more prone to human
mistakes.This motivates our approach to provide a simple yet efficient 
solution to automatically deduce execution policy based on
extra information solelly added on the goals, namely {\em urgency}.
\fcomment{I have nothing to cite here: it is just feedback from experience...}
% As autonomous agents become more and more flexible it becomes
% difficult to deal with all the possible changes that occur during the
% mission. During our work we identified that, while many existing
% architectures embedding a planner do provide solutions to accept goals
% on the fly, the way they handle the execution of the plan beforehand
% is not necessarily fitted for efficient integration of such
% unpredicted situation.

% Until now the only way we could improve this
% was to engineer extra information within the domain model which added
% complexity both in term of the planning search and more importantly
% during the design of such model. This added complexity increase the
% risk of fault introduction within the model, which motivates our
% approach to provide a simple yet efficient solution to deduce
% execution policy based on extra information on each goal {\em
%   urgency}.

The current work is focusing on the plan structure at a fairly
abstract level without direct consideration in our approach of the
timing constraints applied to each goals of the plan. In this approach,
we are able using a breadth first search within the existing plan
structure to select different execution policies for each action
within the plan. By doing so, we can improve the overall mission 
execution as it avoids being proactive on objectives that could lead it
to inefficient handling of new goal requests.

At this stage, our causal link traversal is agnostic of the temporal 
constraints applied to the timepoints of the plan. While it is
beneficial for the generality of the approach, we also see that
including such information could greatly increase the quality of our solution. 
One specific point we see is that often the goals we necessarily mark
as non urgent are tied to a timepoint that is constrained to be {\em
  after} a future date. We need to explore this further in order to
see how such information could be used for our system to automatically
classify such goal as non urgent. It can also impact how the
{\em urgent} goal propagate within the plan. For example, in our result
we had a non urgent goal around 1:00 pm that was eventually 
followed by an urgent goal as a result of the planning. As of today, 
our policy did make all the actions related to the non urgent goal 
become proactive. This can lead to premature surfacing that is then 
prohibiting the agent to accept goals in the morning. Our current
understanding is that this timing constraint enforcing the non
urgent {\em Surface} to end after 1:00 is acting as a guard on the
propagation of our heuristic. Therefore, we should have ideally only
marked as urgent the actions that are directly after this timepoint
instead of continue to propagate to the actions leading to the
surface. 

Such refinement can be done only by considering the temporal
constraints applied to the plan timepoints. Indeed should the 
{\em Surface} not have been constrained to be around 1:00 pm staring
its execution proactively would have not blocked the vehicle at the
surface. We plan, in the future, to explore how both the
marking of the goals {\em as urgent} and propagation to action can be
better deduced by analyzing the simple temporal network (STN)
supporting the current plan. Further, as the planner we used did not
implement STNU and the work presented in \cite{morris01} about dynamic
controllability. We believe that as this work is handled during
planning while our approach is handled during plan execution, both
should be easily complementary still the {\bf wait} actions inserted
by their algorithm may need to be considered with a non
urgent status within our approach. 

All of these refinements are considered in order to further improve
our approach in the context of temporal planning. They would allow
further improvement in the overall execution of our agent despite the fact
that the set of goals it will receive is not fully known until mission 
completion. In our domain such requirement became crucial as our $AUV$
is evolving on a highly dynamic environment and scientists often need
fresh information in order to identify what could be done next. Until
now the common approach on AUV surveys was to separate exploration and
exploitation into two different surveys \cite{Yoerger01012007}. We do
believe that by improving how the vehicle can handle the reception of
new goals we can greatly improve the efficiency of such surveys and can
also generalize to any domain where a robot can receive unexpected
new objectives from its users during the course of its mission.





% While there has already been work around the topic of robust
% \fcomment{This is just a quick presentation on potential future
%   direction} As we stated in the related works our approach does not
% really address dynamic controllability and has the more classic
% assumption present in many planning frameworks that time-points are
% controllable. A side effect of this is that in its current state it
% may result on the system to decide to defer action as late as
% possible. In our example, this would result on the AUV leaving Vent1
% as late as 19:00 making the rest of its plan brittle to any delay due
% for example to downward water currents on its way to the surface. This
% needs to be further addressed in the future and, especially, how our
% work can be integrated with work presented in \cite{morris01}.

% Further as of today, we consider that the qualification of the goal is
% predefined when the goal is submitted by the planner. It is possible
% though that part of this can be refined on some cases based on the
% nature of the goal. Looking back at our domain, one can note that the
% 2 goals provided are constrained differently on their start time;
% while {\em Sample Vent2} start time is limited only on its upper
% bound, the returning to surface conversely is constrained only on the
% lower bound of its start time. This difference hints on some of the
% issues we presented. While we do consider that explicit information of
% these goals help the plan execution to be improved when such
% information is not initially present. We also are aware that the
% nature of the constraints within the goal itself can help identify the
% best policy to be done. It is obvious that for Sampling Vent2 it is
% better to be proactive on the actions that contribute to this
% goal. Conversely, the other goal only matter if it appears fairly late
% in the plan which means that it is probably better to not start
% completing this part of the plan too aggressively. We plan to further
% explore how we can refine the distinction between the different
% policies by using the information provided by the constraints of the
% different objectives.
 
%%% Local Variables: 
%%% mode: latex
%%% TeX-master: "aaai13"
%%% End: 


%\twocolumn 
\bibliographystyle{aaai}
\bibliography{references}

\end{document}

\section{Previous Approaches to Dispatching}

\fcomment{I need to rework this as it is not really a state
  of the art in the current form ... we also need to present that fact
  that not that much people do consider postponment of actions
  ... except maybe for dynamic controlability problem ... this may be
  the key for this part. WE can present proactive approaches as being
  the general agreement except for the special case of dynamic
  controlability where 3DC+  introduces wait actions which in fact put
  timepoints that need to be deferred. We on the other hand try to
  look at another aspect that may trigger the decision to defer an
  actions based on its purpose}

Dealing wiht plan execution s not a new problem and we can find a lot
of work that relates to this during the last decades. Still, it is
pretty rare to see work athta tries to envisage that actions could be
postponned except when this is necessary to not break the current
plan. 


The most prominent work is related to the dispatchability of simple 
temporal networks (STN) \cite{mus98}. The core of the problem is to
ensure that the temporal constraints can be propagated efficiently
within the plan in order to allow the executive to decide quickly
whether an action should  be started or not while ensuring the plan
consistency. In order to accomplish such a task, the STN supporting
the plan to be dispatched is transformed into a All-Pairs network and
stripped of unnecessary edges, resulting often on a more compact
temporal network and lessen the propagation cost of its updates. The
role of the executive is to select timepoints within the current
execution bounds and propagate its value within the simplified
network. Still, while this work contribute to ensure that execution
time are correctly propagated within the plan with a limited cost, it
does not directly address how to decide what value should be set for a
given timepoint in the scope of its possible values. more specifically
it is still the role of the executive to decide whether it should
start an action as early as possible or consider it as not urgent. 




% Dispatchability for a simple temporal network (STN) has been identified
% with whether a plan is capable of fast execution, retains its
% flexibility despite uncertainty in the plan, and all information is
% available at execution \cite{mus98}.  To accomplish the task of
% efficient execution, the STN's are transformed into All-Pairs networks
% and striped of their unnecessary edges, which will often create a more
% compact temporal network and esse the propagation of its updates.
% The executive then selects a timepoint, within the bounds of the 
% execution time, and every timepoint before it has
% already been executed. It then sets the start time to the selected
% timepoint and propagates the needed changes throughout the graph
% \cite{mus98}. The process of selecting a timepoint can lead to
% issues when two timepoints have the possibility of being executed at the
% same time, where one should necessarily come first, but the incorrect
% one gets selected first. This mistake can cause the STN to become
% brittle, which leads to a higher chance of failure in the plan.

\fcomment{Need to continue to refine from here and talk about STNU and
  morris01 as  the only approach we know of where postponed events are present
  though the inserted wait}

In \cite{mus97}\fcomment{Missing reference or is itr pell97 ?}, the execution of events follows the same
properties as before.  The executive is defined as having two jobs the
selecting and scheduling of events for execution. The executive needs to
be highly reactive as it is necessary to function in a real-time
environment. One solution offered for dispatching events efficiently
is the proactive approach. This approach greatly reduces the plan
flexibility, and therefore robustness, as all start timepoints are
grounded to a specific value which is compatible with the inital
constraints. Take our shopping example, and apply the proactive
approach globally.\fcomment{figure again}.  We'll say the agent wakes up
at 9 AM and the duration for traveling to the store is forty minutes
and buying the apples is twenty minutes. Using the proactive approach, the
agent will start shopping as soon as possible and thus be done shopping
by 10 AM. The agent will then go directly home afterwards to satisfy the
``home by 8 PM'' goal. However by doing so, the agent will be back
home by 11am with the constraint to stay at home for the next 9
hours. 
\fhighlight{If his wife is then calling to ask him to buy extra things it will
results on hime to need to fully relax his current plan and go back
and forth between its home and the store all other again. By being
fully proactive the agent.}\fcomment{Maybe what was meant initially
here is more related to dynamic controllability which mean that I need
to twek this text to be more related to a modelled
but uncontrolbale situation.}


% However, the start time for this goal will be roughly
% 11 AM, and if the agent wants to get some dessert before dinner it will
% cause a failure in the plan. What this example shows is how EST can
% severely reduce the flexibility of a mission. EST can also cause
% inconsistency within the plan.

 \fhighlight{Therefore,} \cite{mus97}\fcomment{See comment before} added
implied links between timepoints allowing the planner to find them
efficiently.\fcomment{I don't get what you mean here}

\fcomment{I am here : need to develop}
The only case we have seen in literature where a timepoint is
considered for postponement is related to the approach proposed in 
\cite{morris01} to deal with dynamic controllability issue.
\fcomment{following will be included in former para}

Another issue with execution is uncertainty in the plan. The above
example illustrates the uncertainty about when the agent should go home
or if the agent will want to buy more food during the mission. A
solution for execution uncertainty is the least-committed method, which
waits until that last moment to execute events. This appoarch keeps the
plans flexibility and it allows the executive to have all of the
available information before making a choice (Block, 2006). However, it
has been argued that the least-committed method is an inefficient
solution because in a real-world environment an agent might idle for
long periods of time (Gallien, 2006). This can severely decrease the
overall performance of the system. Lets demonstrate the least-committed
policy in our Shopping agent example using the same goals and durations
as before. This time around the agent won�t want to do anything until it
is absolutely necessary to do it. The agent will then wait until 6:20 PM
before leaving and will finish shopping at 7:20 PM this way the agent
can be home by 8 PM. However, even though this keeps flexibility in the
plan there are a few issues. If there happens to be traffic then the
agent might not make it home by eight PM. Also the agent procrastinates
all day just waiting to go shopping rather than starting early on in the
day, which wastes valuable time. Their (Gallien, 2006) solution is to
change the planning heuristic altering how their plan is created
inevitably adjusting it for faster execution.

Therefore, one reoccurring problem of dispatchability is finding a way
of balancing efficiency, EST, with flexibility, least-committed. There
have been many uses of these two approaches when dispatching plans, as
previously shown, but a compromise to one is most often used rather than
a balance. Demonstrated in the tool MAPGEN (Bresina, 2003) which used
the Earliest Time Solution (ETS) for generating and displaying plans
very quickly. Then allowing the users to manipulate the plan afterwards,
getting around the problem of EST generating undesirable plans. What is
needed is a way of using both EST and least-committed together giving a
balanced approach to dispatching.

%%% Local Variables: 
%%% mode: latex
%%% TeX-master: "aaai13"
%%% End: 

\section{Introduction}
\label{sec:intro}

\fcomment{Intro is too long should be less than a page (1.5 column) and is
  currently more than one. Also we need to speak more about 
  execution task span and less about planning itself}
Substantial research effort in robot planning has revolved around how
plans are executed and how threats to robust execution can be dealt
with in-situ \kcomment{citations needed}. This is particularly
important in dynamic and unpredictable real-world environments where
plans can and do change rapidly and the robotic agent is expected to
be responsive to such exogenous change. Often this focus has been
driven by the need to replan given a fixed set of goals a priori,
where such objectives are known and do not perturb the planning and
execution process.

In this paper, we argue that the impact of asynchronous arrival of
goals and their consequent impact on execution is important 
in dynamic and unpredictable environments especially during robotic
exploration. Our robotic domain deals with benthic as well as upper
water-column exploration using autonomous underwater vehicles (AUVs),
which are untethered and propelled. For instance, in bathymetric
surveys of subduction zones, AUVs are used to augment limited ship
time to deal with largely unexplored areas of coverage. Even so
finding hydrothermal vents is challenging given large spatial areas,
noisy sensors and low signal/noise ratio as also the harsh environment
in which the AUV has to survive. Identifying vents is painstaking;
inadequate sensors and lack of closed loop bathymetric control,
requires scientists on ship analyse previous surveys to inform the
current AUV deployment. This often means that the vehicle can be
re-tasked with new goals while executing one or more current mission
objectives.  AUV deployment in such cases often occurs over multiple
phases \cite{Yoerger01012007}; to minimize station time on site, the
AUV is redeployed after each phase after a vehicle recharge while data
analysis from the past survey continues. Consequent analysis then
provides the impetus to retarget when new scientific targets are
generated on ship.

While substantial literature exists to deal with agent reactivity and
replanning in the face of execution uncertainty \kcomment{citations
  needed}, the ability of a robotic agent to systematically deal with
dynamic goal is less well studied. In this context, we emphasize the
need for planning and plan execution in-situ to enable tight closure
of control loops for the robot's responsiveness.
% In order for an agent to operate in a dynamic environment, the agent
% needs to be constantly observing the external world so that it can
% keep up to date with how the world is changing. Therefore, a robotic
% agent needs to be planning in situ so that it can alter, or update,
% its plan according to the changes. The actual execution of the plan
% will also need to change accordingly. The execution also provide
% constant feedback on how the world is evolving.  Thus as a result of
% the dynamic environment, the robotic agent will need to anticipate
% constant changes in the plan and need to evaluate how to properly
% execute the plan.
Doing so, leverages the ability to dynamically add new objectives to
the agent during mission execution instead of being forced to specify
them beforehand and is the context of this paper.  While such
capabilities allow for more flexible and autonomous missions it also
makes the overall misson execution more challenging.

In the pursuit of planning in a dynamic environment, earlier work done
in \texttt{IPEM} \cite{AmbrosIngerson88}, \texttt{ROGUE}
\cite{Haigh98}, and the LAAS Architecture \cite{alami:1998p820} have
all helped to make real-world planning possible. Based on these past
successes, there have been multiple experiments that have successfully
functioned in the real-world, the Remote Agent Experiment \texttt{RAX}
\cite{mus98}, the Autonomous Spacecraft Experiment \texttt{ASE}
\cite{chien99} and more recently the Teleo Reactive Execution \rxe
\cite{mcgann08b,py10}. In all these systems, generative plans are
dispatched for execution using rich representations that deal with
durative actions and resources \cite{lemai-chenevier2004}, the focus of our work
here.

In particular, our focus is on constraint-based temporal planning
\cite{frank2003,lemai04} with a demonstrated capability for real-world
planning and execution.
% Most of these modern systems are using at the high-level 
% constraint-based planner with rich representation of resources 
% and time such as the ones presented in \cite{frank2003,lemai04}. 
Most of these approaches often implement  least-commitment 
planning and generate a solution that does not commit to specific 
values (\eg action start and end
times) unless explicitly required by the model. Such flexibility puts
the burden on the plan executive to decide at which time within all
possible values, a specific action can be  dispatched. 
% Resolving this problem resulted on a large amount of research both
% considering how to represent the plan to be dispatched to the
% executive with a compact yet complete representation, along with more
% refinements on how to deal with uncontrollable constraints. Still,
The challenge is deciding whether the next possible set of action(s)
should be started immediately or if it should be postponed for a later
start date.

Starting actions as early as possible avoids procrastination and tends
to reduce plan brittleness. This choice
is motivated by the intuition that starting actions early provides more
room in the future to deal with worst case execution scenarios. While
this solution works in the classical case where all the goals of the
agent are known a priori, it can be problematical when new objectives
are added during the course of mission execution. For example, the
agent may prematurely leave an area before new objective for this area
are formulated. On the other hand, waiting inordinately could wast
time and leave less flexibility for coping with diverse execution
scenarios. 
% the agent in a
% situation where the new goal received is not integrated as efficiently
% as it would have been should the agent have waited before taking
% action.
In this context, it is important for the agent to make a distinction
between actions that can be taken proactively versus other actions
that may not be urgent.  

In this work, we propose a systematic approach to allow the agent to
make such a distinction for each action by tracing the causal relations
between goals within the current plan.  This allow us to determine the
best dispatch strategy based on the nature of the goals this action
contributes to.

% \fcomment{Plan for the paper here}
The structure of this paper is as follow. We introduce an example
illustrating the problem of flexible plan execution. We then discuss
previous works and how they tend to focus on a single policy or not
consider the possibile emergence of new goals during the mission. We
then present algorithmic solutions that allow to decide at execution
when actions should be started. Finally, after presenting the results,
we conclude and discuss potential directions for future research.



% this can have a
% negative impact when consider that new goals for the agent could
% emerge as it execute its mission.

	
% \fcomment{I need to rework the following section, started here}
% At the highest level of abstraction, constraint-based temporal
% planning uses a temporal plan to organize how the world works. A
% temporal plan is made up of multiple timelines which each describe the
% evolution of a subsystem in chronological order. An example of a
% subsystem would be the navigator for a robotic agent. Each timeline is
% simply the encapsulation of all the possible state values for a
% subsystem, within the parameters of a plan.  Therefore, a timeline can
% be broken down into its state values along with their temporal
% extents, which are defined as tokens. As the base unit for a timeline,
% tokens describe the simplest behavior of a subsystem e.g. turn
% right. THis kind of representation as it follows least commitment
% planning strategy. It herefore allow to have a plan which is partailly
% grounded and especially tokens star, duration anfd end times are
% represented as intervals depiciting all the acceptable values for
% each. By doing so this leave  mroe flexiblity during execution to
% decided when -- within the specified domains -- an event should
% occur accroding to the current context. This also raises the probelem
% that thids decision should then be taken after the plan was produced
% by the executive. A lot of exitiing work ahas been made in the area on
% the best approach to send the plan to dispach the produced plan to an
% executive so it is easier for it to maintin the temporla constraints
% within this plan at execution (references). This still leave an issue
% open : how does one decide at execution whether a plan actitivy (or
% token) should be started as soon as possible or can be postponed
% later. This problem can become imporatnt when you see the plan being
% produced within a situated agent. Indeed, while the current set of
% objectives fr the agent is known these can evolve as time advance and
% the poperators or other agents give it new objectives to be
% fullfilled.
% It becomes therefore important for the agent to be
% proactive when required while being able to disitinguish actions that
% are not yet urgent.
% It is understood that grounding the start and end times of a
% token decrease the robustness of a robotic agent while in a dynamic
% environment (citations). Therefore, the tokens have flexible intervals
% for their start, duration, and end times.  As an example, the turn
% right behavior could start from a lower bound time of 12pm to an upper
% bound time of 3pm. This means it can start anytime between its lower
% and upper bound, keeping it flexible.  Consequently, it is the agents
% task to choose exactly when to execute such behaviors. We define the
% task of choosing a token and sending it to its appropriate subsystem,
% to be executed, as dispatching.


%%% Local Variables: 
%%% mode: latex
%%% TeX-master: "aaai13"
%%% End: 

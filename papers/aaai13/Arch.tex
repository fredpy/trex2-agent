\section{Executing a flexible temporal plan}
\label{sec: defs}

The focus of this paper is on the possibility of accommodating \emph{new
  objectives} arising during mission execution. 
To preserve context for their inclusion and subsequent execution, and
that the respective policy be applied to each action. Our robot generates and
dispatches a partial plan in parallel with a lower level controller
which further decomposes these plans towards atomic actions. These
actions ultimately actuate hardware on our AUV. The world responds by
sending sensory feedback as \emph{observations} up thru the
hierarchical controller \cite{mcgann08bdup,rajan12dup}. Details of the
controller are beyond the scope of this paper and are not relevant to the
concepts described. 

We first consider some basic representational definitions related to a plan
consistent with \cite{Nau:2004}. A plan has a set of operators and states.
Operators link a set of states, conditions, to a new set of states, effects. We 
refer to this link as a causal link in the plan. Recursive causal links build a path
in the plan from an initial state into a goal state. The plan structure is very similar to the
STRIPS model in defining a plan. 

Within the plan, each state has a start and end time. We refer to
these times as timepoints. Each timepoint is a flexible interval of time $[t, T]$, which
includes all possible values between $t$ and $T$.  This type of plan is referred to as 
a flexible temporal plan. Further, we assume that the
plan, dispatched to us, is in its dispatchable form as defined by \cite{mus98a}.  This leaves an issue 
for execution in determining when a timepoint should start within its flexible interval.

In order to handle this issue, we assume that some states have a defined execution policy. 
A policy can be, but not limited, early start or late start as referred to in our example.  Also the
execution policies are determined beforehand by a user. We then use the policies to 
to determine when to execute a state. Typically, the states that have a defined
execution policy are the goal states of the plan. This leaves states without an execution policy
in which case we use a default policy. In our example, we use a strict latest start time but any policy can 
be used. 


%\begin{definition}
%  \label{def:domain}
%  A temporal planning domain is a triple $D = ( \Lambda_\Phi, O, X )$, where:
%  \begin{itemize}
%  \item $\Lambda_\Phi$ is the set of all temporal databases that can
%    be defined with constraint, constant, variable, and relation
%    symbols in the representation.
%  \item $O$ is a set of temporal planning operators.
%  \item $X$ is a set of domain axioms.
%  \end{itemize}
%\end{definition}

%A temporal plan is a partially order graph of tasks where nodes
%are defined in metric time.
%Each task consists of when it starts and ends or its start
%and end timepoint in the plan. A \emph{token} is a task that holds over the 
%flexible temporal interval consisting of a start and end timepoint.

%A \emph{token} is a property that holds within a flexible temporal
%interval which represents an action or a state associated with $O$ or
%$X$ \cite{py10dup}. A temporal interval is delimited by two {\em
%  timepoints} \cite{Boddy93} marking the possible values of its {\em
%  start} and {\em end}.

%\begin{definition}
%  \label{def:problem}
%  A temporal planning problem in $D$ is a tuple $P = ( D, \Phi_{obs}, \Phi_g )$, where:
%  \begin{itemize}
%  \item $\Phi_{obs} = (F, C)$ is a database in $\Lambda_\Phi$ that
%    satisfies the axioms of $X$.  $\Phi_{obs}$ represents an initial
%    scenario that describes initial state of the domain and
%    observations.
%  \item $\Phi_g = (G, C_g)$ is a database that represents the goals of
%    the problem as a set $G$ of tokens together with a set $C_g$ of
%    objects  and temporal constraints   on variables of $G$.
%  \end{itemize}
%\end{definition}

\begin{definition}
  \label{def:plan}
  A plan is a set $\pi = \{a_1,...,a_k\}$ of actions, each being a
  partial instance of some operator in $O$.  We define $\lambda$ as
  the state-transition function.  $\pi$ is a solution of a problem $P
  = (D, \Phi_{obs}, \Phi_g)$ iff there is a database in $\lambda(\Phi_{obs},
  \pi)$ that entails $\Phi_g$.
\end{definition}


 
While the focus is on executing plan $\pi$, to reflect the dynamic
interaction of the agent with the environment we need to refine the
definition of the sets $\Phi_{obs}$ and $\Phi_g$. As the world
evolves, new (or refinement of existing) observations are added to
$\Phi_{obs}$. Similarly, the agent operator can request new goals in
the future to be added to the agent's $\Phi_g$ as mission time
advances. For the sake of simplicity we consider that the alteration
of these sets is purely additive with time\footnote{In practice, their
  evolution is more complex as completed goals becomes observation;
  even a goal previously requested can be cancelled by the
  operator. While this assumption largely simplifies problem
  description, our algorithm works regardless.}:
\[ \forall \{t, t'\}: t \le t' \Rightarrow \Phi_{obs}(t) \subseteq \Phi_{obs}(t')
\wedge \Phi_g(t) \subseteq \Phi_g(t') \] 

where $\Phi_{obs}(t)$ and $\Phi_g(t)$ are the value of these sets at
time $t$.

$\Phi_{obs}$ grows dynamically, reflecting cumulative observations as
$\pi$ is executed. In nominal situations, new elements of $\Phi_{obs}$
are refinements of the plan -- for example by asserting that a planned
command has just started\footnote{This precludes situations where new
  observations invalidate the plan; they don't refine the plan, but
  invalidate it.}. The agent can also receive new objectives at any
time which are added to $\Phi_g$. 

This assumption has an impact on how it is preferable to handle plan
execution. While deciding when to start a token within the plan, one
needs to ensure that the token execution will not limit the ability of the
agent to treat emergent goals. In light of this, the agent should
attempt to balance the impact of the next available action to execute
as early as possible or to delay it with the eventuality new goals
might occur. Going to \emph{Vent2} early is viable; however, going
back to the surface too early would result in blocking the AUV --
within its current plan -- at the Surface until 20:00. The solution
which provides the most flexibility is for the agent to alternate
between the two policies.

The nature of the goals are better characterized by inclusion of their
priority. For that purpose, we define $\Phi_g$ to be partitioned in two:

\begin{itemize}

\item \emph{internal} goals $\Phi_{gi}$ which represent goals the
  agent {\em needs} to maintain internally. These goals will be
  considered as objectives that can be deferred during execution.

\item \emph{external} goals $\Phi_{ge}$ which represents the goal
  received by the agent from an external source, like a user. These are
  considered to be of higher importance -- \ie the agent {\em wants}
  to execute them. Therefore, their execution should preferably be
  proactive.

\end{itemize}

At any time $t$ during action evaluating by the executive within
$\pi$, the plan is an up to date and valid solution to the set of
goals in $\Phi_g$.

% At any point we need to evaluate an action within plan $\pi$ we
% consider that this plan is up to date and provide a solution of all
% the goals of both $\Phi_{gi}$ and $\Phi_{ge}$ that can reasonably be
% done within the current mission scope.

\section{Algorithmic Description}
\label{sec:algo}

As a new action is to be dispatched for execution, the executive needs
to evaluate how it relates to the goals of the plan. Intuitively, if
this action was generated by (or contributes to) an external goal of
$\Phi_{gi}$ it needs to be taken proactively. Therefore, while
evaluating the token representing this action within $\pi$, the
executive needs to forward search on the causal links related to this
token to see if it leads to an external goal.

% as implemented in
% Algorithm \ref{DispatchToken}.

\begin{definition}
  \label{def:subgoalLink}
  A goal $g$ is causally linked to a state value $c_i$ if
  $g$ is an effect of an action $a$ and $c_i$ is a condition of $a$.
  \footnote{For metric temporal plans, Allen Algebra
    \cite{allen84} is the basis for determining relationships between
    actions and is more general than pre-conditions in classical
    planning.}.
\end{definition}

Consequently, all conditions of this action can be assumed to be
subgoals with a similar policy as $g$.

\begin{algorithm} [H]
  \caption{\small The function $DispatchToken$ finds if there is a
    goal in $\Phi_{ge}$ that is connected to the token, $t$, and, if
    so, dispatches the token.}
  \label{alg:dispatch}
\label{DispatchToken}
\begin{algorithmic}
\small 
\Function{DispatchToken}{Token $T$}
\State Boolean $Goal = \textsc{SearchForGoal}( T )$
\If{ $Goal$ $=$ True}
	\State \Return Dispatch $T$
\ElsIf{ $T$ start upper bound $ \leq $ current tick}
	\State \Return Dispatch $T$
\Else
	\State \Return Don't dispatch $T$
\EndIf
\EndFunction
\end{algorithmic}
\end{algorithm}

Algorithm \ref{alg:dispatch} is key in determining how a token should
be dispatched and is called at every execution cycle while evaluating
the next set of tokens to be executed.  % Finding a token is connected
% to a goal in $\Phi_{ge}$, we conclude that the token is a sub-goal,
% and thus dispatch it proactively.  Conversely, if the token is not
% connected to a goal then we defer dispatching until necessary.  This
% demonstrates our distinction between how we dispatch tokens,
% proactive or deferred.

\begin{algorithm} [htb]
  \caption{\small The function $SearchForGoal$ does a Forward search
    looking for a token that is in the set $\Phi_{ge}$.}
  \label{SearchForGoal}
\begin{algorithmic}
  \small
  \Function{SearchForGoal}{ Token $T$ }
  \If{$T \in \Phi_{ge}$}
  \State \Return True
  \Else 
  \State List $Fringe$ = \{ Actions that $T$ is Condition of \}
  \ForAll{Action(s), $A$, in $Fringe$ }
  \ForAll{Effects, $E$ of $A$}
  \If{$E \in \phi_{ge}$}
  \State \Return True
  \Else 
  \State List $L$ = \{ Action(s) that $E$ is Condition of \}
  \State $Fringe$ = $Fringe \cup L$ 
  \EndIf
  \EndFor
  \EndFor
  \EndIf
  \State \Return False
\EndFunction
\end{algorithmic}
\end{algorithm}


% Algorithm \ref{SearchForGoal} is used to search for a causal link
% between the token $T$ and token $G$ of $\Phi_{ge}$ using
% forward-search. If $G$ is found, then token $T$ is part of the
% solution for completing $G$.  As such, we want to be proactive in
% order to dispatch $G$ as early as possible.  If a $G$ is not found,
% then token $T$ has no connection to an external request and therefore
% there is no motivation to execute this action earlier than
% necessary. This algorithm is equivalent to a breadth first search
% along the planning structure starting from the action we need to
% evaluate. Its complexity is $O(N+E)$ \cite{corman} where $N$ is the
% number of tokens within the plan and $E$ the number of causal links
% that connect these tokens.

To inform this decision we use Algorithm \ref{SearchForGoal} which
searches for a causal link between the token $T$ and $G$ of
$\Phi_{ge}$ using forward-search. If $G$ is found, then token $T$ is
part of the solution for completing $G$. As such, we want to be
proactive in order to dispatch $G$ as early as possible. If a $G$ is
not found, then token $T$ has no connection to an external goal and
therefore there is no motivation to execute this action earlier than
necessary. Its complexity is $O(N+E)$ \cite{corman} where $N$ is the
number of tokens and $E$ the number of causal links that connect these
tokens in the plan. An alternate version of Algorithm
\ref{alg:dispatch} that propagated goal urgency through causal links
during planning was also attempted. While functionally equivalent, the
overhead during backtracking -- requiring removal of propagated
information during search -- along with continuous plan refinement
process in our system, did not demonstrate any significant performance
benefit for evaluation.

% any significant benefit in term of performance and could ben create
% important extra cost should the search for plan been misleader by the
% heuristic.


%\subsection{Dynamic solution during planning}
%
%Searching for a goal as Algorithm \ref{SearchForGoal} does can be
%quite computational expensive particularly if there are many tokens
%that are continuously being dispatched. Completing a full search every
%time a token needs to be dispatched can severely slow down the
%execution process, which needs to remain quick to ensure proper
%execution. Therefore, our next algorithmic approach distributes the
%full search within the creation of the plan.  Resulting in spreading
%out the full cost of the search. In order to not repeatedly search the
%plan, we save the tokens that are connected to a goal in $\Phi_{ge}$
%found during the search. In this way, we acquire a list of tokens,
%$List_{goals}$, that should be dispatched early.
%
%An alternative solution is to embed the propagation of these value
%during the planning search. The algorithm uses the same dispatching
%method as algorithm \ref{DispatchToken}.  The difference is that
%rather than searching for the goal using algorithm
%\ref{SearchForGoal}, it only searches the list, $List_{goals}$, to see
%if the token is in it. The actually searching for the goals and
%causally connected tokens happens in algorithms \ref{NotifyActivated},
%\ref{NotifyMerged}.
%
%\begin{algorithm}[H]
%\caption{\small Saves goals as they are added to plan}
%\label{NotifyAdded}
%\begin{algorithmic}
%\Function{NotifyAdded}{ Token $T$ }
%\If{$T$ is a Goal in $\Phi_{ge}$}
%	\State Insert $T$ into $List_{goals}$
%\EndIf 
%\EndFunction
%\end{algorithmic}
%\end{algorithm}
%
%\begin{algorithm} [H]
%\caption{\small Removes the token after it is removed from the plan}
%\label{NotifyRemoved}
%\begin{algorithmic}
%\Function{NotifyRemoved}{ Token $T$ }
%	\State Remove $T$ from $List_{goals}$
%\EndFunction
%\end{algorithmic}
%\end{algorithm}
%
%\begin{algorithm}[H]
%\caption{\small Searches for tokens connected to goals}
%\label{NotifyActivated}
%\begin{algorithmic}
%\Function{NotifyActivated}{ Token $T$ }
%\If{$T$ is a goal in $\Phi_{ge}$ or $T$ is linked to a goal through one causal link}
%	\State Recursively search the reverse causal link and add the tokens into $List_{goals}$
%\EndIf
%\EndFunction
%\end{algorithmic}
%\end{algorithm}
%
%\begin{algorithm}[H]
%\caption{\small Deactivates token in plan}
%\label{NotifyDeactivated}
%\begin{algorithmic}
%\Function{NotifyDeactivated}{ Token $T$ }
%\If{$T$ is not a goal and not one causally linked to a goal}
%	\State Remove $T$ from $List_{goals}$
%\EndIf
%\EndFunction
%\end{algorithmic}
%\end{algorithm}
%
%\begin{algorithm}[H]
%\caption{\small Searches plan when tokens are merged}
%\label{NotifyMerged}
%\begin{algorithmic}
%\Function{NotifyMerged}{ Token $T$ }
%\If{$T$ is goal in $List_{goals}$}
%	\State Recursively search the reverse causal link of the active token merged with $T$ and add tokens to $List_{goals}$
%\ElsIf{The active token of $T$ is in $List_{goals}$}
%	\State Recursively search the reverse causal link of $T$ and add tokens to $List_{goals}$
%\EndIf
%\EndFunction
%\end{algorithmic}
%\end{algorithm}
%
%\begin{algorithm}[H]
%\caption{\small Removes token when split}
%\label{NotifySplit}
%\begin{algorithmic}
%\Function{NotifySplit}{ Token $T$ }
%\If{$T$ is not a goal or not one causally linked to a goal}
%	\State Remove $T$ from $List_{goals}$
%\EndIf
%\EndFunction
%\end{algorithmic}
%\end{algorithm}
%
%In order to distribute the search, we situate our algorithm within the
%planning search of the Europa Planner,\cite{frank2003}, which offers 
%callback functions for when a token in the plan is altered. The majority
%of the searching happens when tokens are either activated or 
%merged. For a token to merge with another token it has to be compatible with
%another token already in the plan. Splitting happens when they are
%no longer compatible. We have designed our algorithm around the
%Europa Planner, however, we believe that the general approach
%can work on any other planner. 
%
%Taking full advantage of the planning search, we
%use a backwards search from the goal following the reverse causal link
%to the connected tokens.  We fully search from the goals because we
%know that all the tokens connected through the causal link are
%sub-goals. By contrast, fully searching each token could be
%wasteful because there is no certainty that it will be linked to a
%goal and, therefore, could bring little value to our search. However,
%some tokens may get added to the plan or linked to a goal after we
%have already searched the goals. Therefore, for every token we do a
%local forward search of one causal link to verify if it is connected
%to a goal in our saved list.  If so, we do a full backwards search
%from the token since it has now proven to be valuable. After the plan
%has been searched, it is as easy as searching a list for a token to
%see if it should be dispatched early or be deferred to later.
%
%This approach is potentially more costly than the previous algorithm 
%as it needs to do local updates whenever the plan is altered by the search
%including retracting past updates if a backtrack occur during the
%search. Still it is compelling in the fact that this cost occurs during
%planning reducing the decision problem during execution to simply
%check if the given action has been marked during planning. It has
%been the solution we have preferred within our system for this reason
%as it is functionally equivalent to previous algorithm while reducing
%extra computation cost as the plan is executed. Planning phases occur
%within the plan only when either the initial plan failed to execute or
%the set of goals has been altered. Therefore, it is safe to assume that
%planning should occur more sporadically than execution decisions 
%which then give an edge to this latter algorithmic solution. 

%%% Local Variables: 
%%% mode: latex
%%% TeX-master: "aaai13"
%%% End: 
